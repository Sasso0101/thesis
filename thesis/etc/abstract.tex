\chapter*{Abstract}
\label{cha:abtract}
\addcontentsline{toc}{chapter}{Abstract}
Breadth-First Search (BFS) is a fundamental algorithm for graph analysis, but its performance on modern multicore CPUs is generally limited by irregular memory access. The non-contiguous memory accesses inherent in traversing large-scale graphs lead to poor cache utilization and high-latency memory stalls, creating a significant performance bottleneck for parallel implementations. This thesis investigates and implements two distinct optimization strategies for parallel BFS, targeting modern energy-efficient multicore architectures and focusing on large-diameter graphs, such as road networks.

The first contribution is a cache-optimized implementation in C++ with OpenMP, which proposes a novel MergedCSR data structure. By co-locating vertex metadata with its adjacency list, this format enhances spatial locality to reduce cache misses. The second contribution is an explicitly parallelized implementation in C with pthreads, which provides fine-grained control over the execution model. This version employs a persistent thread pool, a chunk-based frontier with dynamic work-stealing for load balancing, and a lightweight, custom barrier for scalable synchronization.

Both implementations were evaluated against the GAP Benchmark Suite on a diverse set of graphs across x86, RISC-V, and ARM platforms. The results demonstrate that the MergedCSR data structure significantly improves memory performance, enabling a geomean speedup of up to 1.5x over the baseline on large-diameter graphs. Furthermore, the explicit pthreads implementation exhibits superior scalability due to its custom synchronization and outperforms the GAP benchmark with a geomean of 2.28x on road networks and 1.87x on random geometric graphs. This work concludes that achieving optimal performance for memory-bound graph algorithms requires a holistic approach: combining cache-aware data structure design with a fine-grained parallel execution model with low-overhead synchronization.